\noindent The following events have been assessed and mitigated as part of the Structures subsystem:

\begin{itemize}
	 \item \textbf{SRV-RISK-STRUC-1} Propellant tank rupture, results in vehicle explosion.
	\begin{itemize}
		 \item Probability mitigation (-1): overdesign structure with safety margins, inspection before every mission.		 \item Impact mitigation (-1): having an abort system .	\end{itemize}
	 \item \textbf{SRV-RISK-STRUC-2} Thrust structure failure, results in possible damage of the tanks, unwanted side thrust, engine damage.
	\begin{itemize}
		 \item Probability mitigation (-1): overdesign structure with safety margins, often inspections.		 \item Impact mitigation (-1): designing for specific failure modes.	\end{itemize}
	 \item \textbf{SRV-RISK-STRUC-3} Pipe leakage, results in lost of the propellant, depressurization of the propulsion system, vehicle explosion.
	\begin{itemize}
		 \item Probability mitigation (-1): plumbing testing before mission, system redundancy.		 \item Impact mitigation (-1): having an abort system .	\end{itemize}
	 \item \textbf{SRV-RISK-STRUC-4} Landing leg collapsing, results in vehicle with tilt angle after landing, possible falling over.
	\begin{itemize}
		 \item Probability mitigation (-1): redundancy in a system, overdesign structure with safety margins, .		 \item Impact mitigation (-2): special crumple zones, designing for specific failure modes.	\end{itemize}
	 \item \textbf{SRV-RISK-STRUC-5} Skirt buckling, results in damage of the structure, load path interruption.
	\begin{itemize}
		 \item Probability mitigation (-1): overdesign structure with safety margins, .		 \item Impact mitigation (-1): designing for specific failure modes.	\end{itemize}
	 \item \textbf{SRV-RISK-STRUC-6} Separation mechanism failure, results in launch abort failure.
	\begin{itemize}
		 \item Probability mitigation (-1): redundancy in a system, inspection before every mission, fail-safe system.	\end{itemize}
	 \item \textbf{SRV-RISK-STRUC-7} Hydraulics system failure, results in propulsion system failure, landing legs failure.
	\begin{itemize}
		 \item Probability mitigation (-1): redundancy in a system, often inspections.	\end{itemize}
	 \item \textbf{SRV-RISK-STRUC-8} Legs not deploying, results in raugh landing, damage of the vehicle, crew abort.
	\begin{itemize}
		 \item Probability mitigation (-1): redundancy in a system, often inspections.		 \item Impact mitigation (-1): special crumple zones in case of rough landing.	\end{itemize}
	 \item \textbf{SRV-RISK-STRUC-9} Decompression of the capsule, results in lack of breathable atmosphere, possible rupture of the capsule.
	\begin{itemize}
		 \item Probability mitigation (-1): overdesign structure with safety margins, leak testing before missions.		 \item Impact mitigation (-1): having spacesuits for crew.	\end{itemize}
	 \item \textbf{SRV-RISK-STRUC-10} Buckling of the capsule, results in structure failure, possible decompression and rupture of the capsule.
	\begin{itemize}
		 \item Probability mitigation (-1): overdesign structure with safety margins, testing before missions.		 \item Impact mitigation (-1): designing for specific failure modes.	\end{itemize}
	 \item \textbf{SRV-RISK-STRUC-11} Attenuation system failure, results in crew injuries.
	\begin{itemize}
		 \item Probability mitigation (-2): redundancy in a system, overdesign structure with safety margins, .		 \item Impact mitigation (-1): designing for specific failure modes.	\end{itemize}
	 \item \textbf{SRV-RISK-STRUC-12} Side hatch not opening, results in only main hatch usable.
	\begin{itemize}
		 \item Probability mitigation (-1): redundancy in a system, inspections before missions, .		 \item Impact mitigation (-1): ability to perform entire mission through main hatch.	\end{itemize}
	 \item \textbf{SRV-RISK-STRUC-13} Docking hatch not opening, results in unable to transport anything through the hatch.
	\begin{itemize}
		 \item Probability mitigation (-1): redundancy in a system, possibility for manual opening.	\end{itemize}
	 \item \textbf{SRV-RISK-STRUC-14} Capsule ring not deploying, results in unable to dock to the orbital node.
	\begin{itemize}
		 \item Probability mitigation (-1): redundancy in a system, possibility for manual deployment .	\end{itemize}
	 \item \textbf{SRV-RISK-STRUC-15} Non-sealed connection, results in depressurization of the system, unable to open main hatch.
	\begin{itemize}
		 \item Probability mitigation (-1): redundancy in a system, repeating docking procedure.	\end{itemize}
	 \item \textbf{SRV-RISK-STRUC-16} Natural frequency oscillation, results in vehicle rupture and explosion.
\end{itemize}

\noindent From this list, a mitigated risk map has been created, and can be seen in \autoref{tab:risk-map-struc-mitig}.

\definecolor{rm-3}{HTML}{FE0000}\definecolor{rm-2}{HTML}{FF9900}\definecolor{rm-1}{HTML}{FCFF2F}\definecolor{rm-0}{HTML}{00FF00}
\begin{table}[H]
\centering
\caption{Risk map of the Structures subsystem, after mitigation}
\label{tab:risk-map-struc-mitig}\begin{tabular}{l|c|c|c|c|c|}
\cline{2-6}
& \multicolumn{1}{l|}{Very unlikely (1)} & \multicolumn{1}{l|}{Unlikely (2)} & \multicolumn{1}{l|}{Possible (3)} & \multicolumn{1}{l|}{Likely (4)} & \multicolumn{1}{l|}{Very likely (5)} \\ \hline
\multicolumn{1}{|l|}{Very high impact (5)} & \cellcolor{rm-3}6 & \cellcolor{rm-3} & \cellcolor{rm-3} & \cellcolor{rm-3} & \cellcolor{rm-3}\\ \hline 
\multicolumn{1}{|l|}{High impact (4)} & \cellcolor{rm-2}1, 3, 7, 11, 16 & \cellcolor{rm-2} & \cellcolor{rm-2} & \cellcolor{rm-3} & \cellcolor{rm-3}\\ \hline 
\multicolumn{1}{|l|}{Medium impact (3)} & \cellcolor{rm-0}2, 5, 8, 9, 10, 13, 14, 15 & \cellcolor{rm-1} & \cellcolor{rm-1} & \cellcolor{rm-2} & \cellcolor{rm-3}\\ \hline 
\multicolumn{1}{|l|}{Low impact (2)} & \cellcolor{rm-0} & \cellcolor{rm-0}4 & \cellcolor{rm-1} & \cellcolor{rm-2} & \cellcolor{rm-3}\\ \hline 
\multicolumn{1}{|l|}{Very Low impact (1)} & \cellcolor{rm-0} & \cellcolor{rm-0}12 & \cellcolor{rm-0} & \cellcolor{rm-2} & \cellcolor{rm-3}\\ \hline 
\end{tabular} 
\end{table}

 \todo[inline]{Discuss the mitigated map, and alter the text if needed.}